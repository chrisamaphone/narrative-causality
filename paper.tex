\documentclass{llncs}
% \usepackage{makeidx}  % allows for indexgeneration

\title{Linear Logic Programming for Narrative Generation}
\author{Anne-Gwenn Bosser \and Chris Martens \and Joao Ferreira}
\date{\today}

\begin{document}
\maketitle

% \frontmatter          % for the preliminaries
% \pagestyle{headings}  % switches on printing of running heads
% \addtocmark{Hamiltonian Mechanics} % additional mark in the TOC

\begin{abstract}
Recent years have seen the widespread adoption of planning techniques for
the construction of narrative generation systems serving as the backbone
of interactive narrative systems. There is however no representational
formalism which clearly takes precedence, as IS has developed mostly
empirically as an application of various planning systems. In an attempt
to return to the core properties of narratives, which may transcend story
genres and narrative generation systems by providing a unifying underlying
framework, Linear Logic has recently been proposed as a suitable
representational model for computational narratives. 

In this paper, we show that Linear Logic Programming can also provide
sufficient generation mechanisms to sustain the development of novel
interactive storytelling systems, as an alternative to planning
approaches. 
\end{abstract}

\end{document}
